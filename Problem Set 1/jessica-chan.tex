\documentclass[10pt]{article}
\author{Jessica Chan}
 
\usepackage{hyperref}
\usepackage[margin=1in]{geometry} 
\usepackage{amsmath,amsthm,amssymb, graphicx, multicol, array}
 \usepackage{listings}
\usepackage[T1]{fontenc}
\usepackage{textcomp}
\usepackage{url}
\usepackage[variablett]{lmodern}

\lstset{
 basicstyle=\ttfamily,
 columns=fullflexible,
 upquote,
 keepspaces,
 literate={*}{{\char42}}1
 {-}{{\char45}}1
}

\begin{document}
 
\title{LING/COMP 445, LING 645\\Problem Set 1}
\date{}
\maketitle


There are several types of questions below. For programming questions,
please put your answers into a file called
\texttt{lastname-firstname.clj}. Be careful to follow the instructions
exactly and be sure that all of your function definitions use the
precise names, number of inputs and input types, and output types as
requested in each question.

To do the computational problems, we recommend that you install
Clojure on your local machine and write and debug the answers to each
problem in a local copy of \texttt{lastname-firstname.clj}. You can
find information about installing and using Clojure here
\url{https://clojure.org/}.

For questions involving answers in English or mathematics or a
combination of the two, put your answers to the question in an
\textbf{Answer} section like in the example below. You can find more
information about \LaTeX\ here \url{https://www.latex-project.org/}.

Once you have answered the question, please compile your copy of this
\LaTeX\  into a PDF and submit (i) the compiled PDF (ii) the raw \LaTeX\
file and (iii) your \texttt{lastname-firstname.clj} via email to
{\em both} \href{mailto:timothy.odonnell@mcgill.ca}{timothy.odonnell@mcgill.ca} and
\href{mailto:savanna.willerton@mail.mcgill.ca}{savanna.willerton@mail.mcgill.ca}. Please make sure all of your
code compiles and runs, we cannot give partial credit for code that
won't run at all.


\hrulefill
\paragraph{Problem 0:}
This is an example question using some fake math like this
$L=\sum_0^{\infty} \mathcal{G} \delta_x$.

\paragraph{Answer 0:}  $L=\sum_0^{\infty} \mathcal{G} \delta_x$.


\hrulefill
\paragraph{Problem 1:}
Write a procedure called \texttt{abs} that takes in a number, and
computes the absolute value of the number. It should do this by
finding the square root of the square of the argument. (Note: you
should use the \texttt{Math/sqrt} procedure built in to Clojure, which
returns the square root of a number.)

\paragraph{Answer 1:} Please put your answer in \texttt{firstname-lastname.clj}.

\hrulefill
\paragraph{Problem 2:}
In both of the following procedure definitions, there are one or more errors of
some kind. Explain what's wrong and why, and fix it:

\begin{lstlisting}
(defn take-square
  (* x x))
\end{lstlisting}

\begin{lstlisting}
(defn sum-of-squares [(take-square x) (take-square y)]
  (+ (take-square x) (take-square y)))
\end{lstlisting}

\paragraph{Answer 2:} The function abs is missing the argument \texttt{[x]} that should be passed.
 The function sum-of-squares' arguments should not have the take-squared function called on them.
 They should look like this instead: \texttt{[x y]}

\hrulefill
\paragraph{Problem 3:}
The expression \texttt{(+ 11 2)} has the value \texttt{13}. Write four
other different Clojure expressions whose values are also the number
\texttt{13}.  Using \texttt{def} name these expressions
\texttt{exp-13-1}, \texttt{exp-13-2}, \texttt{exp-13-3}, and
\texttt{exp-13-4}.

\paragraph{Answer 3:} 
\begin{lstlisting}
(def exp-13-1 []
	(* 13 1))
(def exp-13-2 []
	(+ 3 10))
(def exp-13-3 []
	(+ (expt 3 2) 4))
(def exp-13-4 []
	(- 15 2))
\end{lstlisting}

\hrulefill
\paragraph{Problem 4:}
  Write a procedure, called \texttt{third}, that selects the third
  element of a list. For example, given the list \texttt{'(4 5 6)},
  third should return the number \texttt{6}.

\paragraph{Answer 4:} Please put your answer in \texttt{firstname-lastname.clj}.

\hrulefill
\paragraph{Problem 5:}
Write a procedure, called \texttt{compose}, that takes two one-place
functions \texttt{f} and \texttt{g} as arguments. It should return a
new function, the composition of its input functions, which computes
\texttt{f(g(x))} when passed the argument \texttt{x}. For example, the
function \texttt{Math/sqrt} (built in to Clojure from Java) takes the
square root of a number, and the function \texttt{Math/abs} (also
built in to Clojure) takes the absolute value of a number. If we make
these functions Clojure native functions using \texttt{fn}, then
\texttt{((compose Math/sqrt Math/abs) -36)} should return \texttt{6},
because the square root of the absolute value of \texttt{-36} equals
\texttt{6}.

\begin{lstlisting}
(defn sqrt [x] (Math/sqrt x))
(defn abs [x] (Math/abs x))
((compose sqrt abs) -36)
\end{lstlisting}

\paragraph{Answer 5:} Please put your answer in \texttt{firstname-lastname.clj}.

\hrulefill
\paragraph{Problem 6:}
  Write a procedure \texttt{first-two} that takes a list as its argument,
  returning a two element list containing the first two elements of
  the argument. For example, given the list \texttt{'(4 5 6)},
  \texttt{first-two} should return \texttt{'(4 5)}.

\paragraph{Answer 6:} Please put your answer in \texttt{firstname-lastname.clj}.

\hrulefill
\paragraph{Problem 7:}
Write a procedure \texttt{remove-second} that takes a list, and
returns the same list with the second value removed. For example,
given \texttt{(list 3 1 4)}, \texttt{remove-second} should return
\texttt{(list 3 4)}

\paragraph{Answer 7:} Please put your answer in \texttt{firstname-lastname.clj}.

\hrulefill
\paragraph{Problem 8:}
  Write a procedure \texttt{add-to-end} that takes in two arguments: a
  list \texttt{l} and a value \texttt{x}. It should return a new list
  which is the same as \texttt{l}, except that it has \texttt{x} as
  its final element. For example, \texttt{(add-to-end (list 5 6 4) 0)}
  should return \texttt{(list 5 6 4 0)}.

\paragraph{Answer 8:} Please put your answer in \texttt{firstname-lastname.clj}.

\hrulefill
\paragraph{Problem 9:}
  Write a procedure, called \texttt{reverse}, that takes in a list, and returns
  the reverse of the list. For example, if it takes in \texttt{'(a b c)}, it
  will output \texttt{'(c b a)}.

\paragraph{Answer 9:} Please put your answer in \texttt{firstname-lastname.clj}.

\hrulefill
\paragraph{Problem 10:}
  Write a procedure, called \texttt{count-to-1}, that takes a positive
  integer \texttt{n}, and returns a list of the integers counting down
  from \texttt{n} to \texttt{1}. For example, given input \texttt{3},
  it will return \texttt{(list 3 2 1)}.

\paragraph{Answer 10:} Please put your answer in \texttt{firstname-lastname.clj}.

\hrulefill
\paragraph{Problem 11:}
  Write a procedure, called \texttt{count-to-n}, that takes a positive
  integer \texttt{n}, and returns a list of the integers from
  \texttt{1} to \texttt{n}. For example, given input \texttt{3}, it
  will return \texttt{(list 1 2 3)}. Hint: Use the procedures
  \texttt{reverse} and \texttt{count-to-1} that you wrote in the
  previous problems.

\paragraph{Answer 11:} Please put your answer in \texttt{firstname-lastname.clj}.

\hrulefill
\paragraph{Problem 12:}
  Write a procedure, called \texttt{get-max}, that takes a list of numbers, and
  returns the maximum value.

\paragraph{Answer 12:} Please put your answer in \texttt{firstname-lastname.clj}.

\hrulefill
\paragraph{Problem 13:}
  Write a procedure, called \texttt{greater-than-five?}, that takes a
  list of numbers, and replaces each number with \texttt{true} if the
  number is greater than \texttt{5}, and \texttt{false} otherwise. For
  example, given input \texttt{(list 5 4 7)}, it will return
  \texttt{(list false false true)}. Hint: Use the function
  \texttt{map} that we discussed in class.

\paragraph{Answer 13:} Please put your answer in \texttt{firstname-lastname.clj}.

\hrulefill
\paragraph{Problem 14:}
Write a procedure, called \texttt{concat-three}, that takes three
sequences (represented as lists), \texttt{x}, \texttt{y}, and
\texttt{z}, and returns the concatenation of the three sequences. For
example, given the sequences \texttt{(list 'a 'b)}, \texttt{(list 'b
  'c)}, and \texttt{(list 'd 'e)}, the procedure should return
\texttt{(list 'a 'b 'b 'c 'd 'e)}.

\paragraph{Answer 13:} Please put your answer in \texttt{firstname-lastname.clj}.

\hrulefill
\paragraph{Problem 15:}
Write a procedure, called \texttt{sequence-to-power}, that takes a
sequence (represented as a list) \texttt{x}, and a positive integer
\texttt{n}, and returns the sequence \texttt{x}$^\texttt{n}$. For
example, given the sequence \texttt{(list 'a 'b)} and the number
\texttt{3}, the procedure should return \texttt{(list 'a 'b 'a 'b 'a
  'b)}.

\paragraph{Answer 15:} Please put your answer in
\texttt{firstname-lastname.clj}.

\hrulefill
\paragraph{Problem 16:}
Define $L$ as a language containing a single sequence,
$L={a}$. Write a procedure \texttt{in-L?} that takes a sequence
(represented as a list), and decides if it is a member of the language
$L^*$. That is, given a sequence $x$, the procedure should return
\texttt{true} if and only if $x$ is a member of $L^*$, and
\texttt{false} otherwise.

\paragraph{Answer 16:} Please put your answer in
\texttt{firstname-lastname.clj}.

\hrulefill
\paragraph{Problem 17:}
Let $A$ and $B$ be languages. We'll use $\textsc{concat}(A,B)$ to
denote the concatenation of $A$ and $B$, in that order. Find an
example of languages $A$ and $B$ such that
$\textsc{concat}(A,B)=\textsc{concat}(B,A)$.

\paragraph{Answer 17:} 

Let $A = {a, aa}$ and $B = {aaaa, aaa}$.
Then $AB = {aaaaa, aaaa, aaaaaa}$.
Which is equal to $BA = {aaaaa, aaaa, aaaaaa}$.

\hrulefill
\paragraph{Problem 18:}
Let $A$ and $B$ be languages. Find an example of languages $A$ and $B$
such that $\textsc{concat}(A,B)$ does not equal $\textsc{concat}(B,A)$

\paragraph{Answer 18:} 
Let $A = {a, ab}$ and $B = {b, ba}$.
Then $AB = {ab, aba, abb, abba}$.
Which is not equal to $BA = {ba, bab, baa, baab} $.

\hrulefill
\paragraph{Problem 19:}
Find an example of a language $L$ such that $L=L^2$,
i.e. $L=\textsc{concat}(L,L)$.

\paragraph{Answer 19:} 
Let $L = {a, ab, aba}$. \\* 
$ L^2 = {aa, aab, aaba, abab, ababa, abaaba}$ and \\*
$LL = {aa, aab, aaba, abab, ababa, abaaba}$. 
We can see the two are equal.

\hrulefill
\paragraph{Problem 20:}
Argue that the intersection of two languages $L$ and $L'$ is always
contained in $L$.

\paragraph{Answer 20:} 
The intersection of languages $L$ and $L'$ is a subset equal to $L \cap L'$ .
Breaking this down, we know that any element $x \in L \cap L'$ must belong in both sets by definition of an intersection.
So, all elements $x \in L \cap L'$ will also be in $L$.

\hrulefill
\paragraph{Problem 21:}
Let $L_1$, $L_2$, $L_3$, and $L_4$ be languages. Argue that the union
of Cartesian products $(L_1 \times L_3) \cup (L_2 \times L_4)$ is
always contained in the Cartesian product of unions
$(L_1 \cup L_2) \times (L_3 \cup L_4)$.
$L' \times L$
\paragraph{Answer 21:} 
We know that any elements in the arbitrary pair $(x, y) \in (L_1 \times L_3) \cup (L_2 \times L_4)$ are either a Cartesian product of $(L_1 \times L_3) , (L_2 \times L_4)$ or both by the definition of union.
\\

\\
From $(L_1 \cup L_2) \times (L_3 \cup L_4)$ we can deduce that any pair $(m,n) \in (L_1 \cup L_2) \times (L_3 \cup L_4)$ will be a cross product of $m \in (L_1 \cup L_2)$ and $n \in (L_3 \cup L_4)$. 
\\
\\
So, we can see that the possible cross products of this union are $(m, n) \in (L_1 \times L_3) \cup (L_2 \times L_3) \cup (L_1 \times L_4) \cup (L_2 \times L_4)$ by the distributive property of union.
\\
\\
We can see that if $(x,y) \in (L_1 \times L_3) \cup (L_2 \times L_4)$, then $(x, y) \in (L_1 \times L_3) \cup (L_2 \times L_3) \cup (L_1 \times L_4) \cup (L_2 \times L_4)$ also since the former is a subset of the latter. Thus the Union of Cartesian products is always in a Cartesian product of Unions.

\hrulefill
\paragraph{Problem 22:}
Let $L$ and $L'$ be finite languages. Show that the number of elements
in the Cartesian product $L \times L'$ is always equal to the number
of elements in $L' \times L$.

\paragraph{Answer 22:} We know that the number of elements in some Cartesian product of $(A \times B)$ is given by $|A| \bullet |B|$. So, by the commutative property of multiplication, we know that the number of elements in $L \times L'$ is equal to the number of elements in $L' \times L$, since $|L| \bullet |L'| = |L'| \bullet |L|$

\hrulefill
\paragraph{Problem 23:}

Suppose that the concatenation of a language L is equal to itself:
$\texttt{concat}(L,L) = L$. Show that $L$ is either the empty set or
an infinite language. (Remember that the empty set contains the null
string.)

\paragraph{Answer 23:} 
Let $L$ be the empty set. We know that concatenating the empty set to the empty set will yield the empty set since sets are unique.
\\
\\ 
Now, let $I$ be the infinite set ${ab}^\infty$. Concatenating this infinite set to itself will yield more of the same pattern ab, however any "new" iterations of this repetition will already be a part of the original infinite set. Since sets are unique and any "new patterns" concatenated are mere pointers to the original instantiation, nothing new is added.  Therefore, concatenation of $I$ to $I$ yields itself. 

\end{document}
